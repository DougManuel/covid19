
          %%%%% ~~~~~~~~~~~~~~~~~~~~ %%%%%

\section{A change point model}
\setcounter{theorem}{0}
\setcounter{equation}{0}

%\cite{vanDerVaart1996}
%\cite{Kosorok2008}

%\renewcommand{\theenumi}{\alph{enumi}}
%\renewcommand{\labelenumi}{\textnormal{(\theenumi)}$\;\;$}
\renewcommand{\theenumi}{\roman{enumi}}
\renewcommand{\labelenumi}{\textnormal{(\theenumi)}$\;\;$}

          %%%%% ~~~~~~~~~~~~~~~~~~~~ %%%%%

Consider the set \,\texttt{coal}\, available in the R package \,\texttt{boot}:
\begin{center}
\texttt{https://stat.ethz.ch/R-manual/R-devel/library/boot/html/coal.html}
\end{center}
which was first published in:
\begin{center}
\begin{minipage}{5in}
A Note on the Intervals Between Coal-Mining Disasters \\
Author(s): R. G. Jarrett \\
Source: Biometrika, Vol. 66, No. 1 (Apr., 1979), pp. 191-193 \\
Published by: Oxford University Press on behalf of Biometrika Trust \\
Stable URL: \texttt{https://www.jstor.org/stable/2335266}
\end{minipage}
\end{center}

\vskip 0.5cm
\noindent
The \,\texttt{coal}\, data frame has 191 rows and a single numeric column (with column name ``date''),
containing the dates of occurrence (in years, with dates represented as digits after the decimal point)
of the 191 explosions in coal mines in the United Kingdom which:
\begin{itemize}
\item
	occurred between March 15, 1851 and March 22 1962, and
\item
	resulted in 10 or more fatalities.
\end{itemize}

          %%%%% ~~~~~~~~~~~~~~~~~~~~ %%%%%

\vskip 0.5cm
\noindent
\textbf{Observed data:}
\begin{itemize}
\item
	The observed data consist of a vector of counts:
	\begin{equation*}
	D \;\; = \;\; (D_{1}, D_{2}, \ldots , D_{T}) \;\; \in \;\; \left(\N\cup\{\,0\,\}\right)^{T}\,,
	\end{equation*}
	where
	$D_{t}$ represents the number of coal mine explosions that occurred during the $t$-th year,
	for $t = 1, 2, \ldots, T$.
\end{itemize}

          %%%%% ~~~~~~~~~~~~~~~~~~~~ %%%%%

\vskip 0.5cm
\noindent
\textbf{Model:}
\begin{itemize}
\item
	User-prescribed hyperparameters:
	\begin{equation*}
	r_{e}\,,\, r_{l} \; \in \; (1,\infty)\,,
	\quad\quad
	T \; \in \; \N
	\end{equation*}
\item
	Parameters and their prior distributions:
	\begin{equation*}
	R_{e} \;\sim\; \textnormal{Exponential}(\,r_{e}\,)\,,
	\quad\quad
	R_{l} \;\sim\; \textnormal{Exponential}(\,r_{l}\,)\,,
	\quad\quad
	S \;\sim\; \textnormal{Uniform}\!\left(\,\{\overset{{\color{white}.}}{1},2,\ldots,T\}\,\right)
	\end{equation*}
\item
	Conditional likelihood of observed data:
	\begin{equation*}
	D_{t}
	\;\; \sim \;\;
		\left\{\begin{array}{ccc}
		\textnormal{Poisson}(\,r_{e}\,)\,, & \textnormal{if \,$t < S$}
		\\
		\overset{{\color{white}1}}{\textnormal{Poisson}(\,r_{l}\,)}\,, & \textnormal{if \,$t \geq S$}
		\end{array}\right.,
	\quad\;\;
	t = 1, 2, \ldots, T
	\end{equation*}
\end{itemize}

          %%%%% ~~~~~~~~~~~~~~~~~~~~ %%%%%

\vskip 0.5cm
\noindent
\textbf{Full joint probability:}
\begin{eqnarray*}
P\!\left(\,D,S,R_{e},R_{l}\,\right)
& = &
	P\!\left(\,D \;\vert\, S,R_{e},R_{l}\,\right)
	\cdot
	P\!\left(\,S,R_{e},R_{l}\,\right)
\;\; = \;\;
	P\!\left(\,D \;\vert\, S,R_{e},R_{l}\,\right)
	\cdot
	P\!\left(\,S\,\right)
	\cdot
	P\!\left(\,R_{e}\,\right)
	\cdot
	P\!\left(\,R_{l}\,\right)
\\
& = &
	\left(\;
		\overset{T}{\underset{t=1}{\prod}}\,
		P\!\left(\,D_{t} \;\vert\, S,R_{e},R_{l}\,\right)
		\right)
	\times\,
	P\!\left(\,S\,\right)
	\cdot
	P\!\left(\,R_{e}\,\right)
	\cdot
	P\!\left(\,R_{l}\,\right)
\end{eqnarray*}

          %%%%% ~~~~~~~~~~~~~~~~~~~~ %%%%%

\vskip 0.5cm
\noindent
\textbf{Marginal joint probability (to be coded in \texttt{Stan}):}
\vskip 0.3cm
\noindent
Since \texttt{Stan} does NOT support categorical latent variables,
the full joint probability cannot be implemented in \texttt{Stan}.
In other to perform inference involving categorical latent variables,
we implement instead the marginal probability,
with all categorical latent variables having been marginalized out.

\vskip 0.5cm
\noindent
In our present example, we marginalize out
\,$S \;\sim\; \textnormal{Uniform}\!\left(\,\{\overset{{\color{white}.}}{1},2,\ldots,T\}\,\right)$:
\begin{eqnarray*}
P\!\left(\,D,R_{e},R_{l}\,\right)
&=&
	{\color{red}P\!\left(\,D \;\vert\, R_{e},R_{l}\,\right)}
	\cdot
	P\!\left(\,R_{e},R_{l}\,\right)
\;\; = \;\;
%& = &
%	\overset{T}{\underset{s=1}{\sum}}\;
%	P\!\left(\,D,S\overset{{\color{white}1}}{=}s,R_{e},R_{l}\,\right)
%\;\; = \;\;
	{\color{red}
	\overset{T}{\underset{s=1}{\sum}}\;
	P\!\left(\,\left.D,S\overset{{\color{white}1}}{=}s \;\right\vert R_{e},R_{l}\,\right)}
	\cdot
	P\!\left(\,R_{e},R_{l}\,\right)
\\
& = &
	P\!\left(\,R_{e}\,\right)
	\cdot
	P\!\left(\,R_{l}\,\right)
	\;\cdot\;
	\overset{T}{\underset{s=1}{\sum}}\;\,
	P\!\left(\, D \;\left\vert\;S\overset{{\color{white}1}}{=}s,R_{e},R_{l} \right.\,\right)
	\cdot
	P\!\left(\,\left.S\overset{{\color{white}1}}{=}s\;\right\vert\,R_{e},R_{l}\,\right)
\\
& = &
	P\!\left(\,R_{e}\,\right)
	\cdot
	P\!\left(\,R_{l}\,\right)
	\,\times\,
	\overset{T}{\underset{s=1}{\sum}}
	\left(\,
		\overset{T}{\underset{t=1}{\prod}}\;
		P\!\left(\,D_{t} \,\left\vert\; S\overset{{\color{white}1}}{=}s,R_{e},R_{l}\right.\,\right)
		\right)
	\cdot
	P\!\left(\,S=s \;\vert\,R_{e},R_{l}\,\right)
\\
& = &
	P\!\left(\,R_{e}\,\right)
	\cdot
	P\!\left(\,R_{l}\,\right)
	\,\times\,
	{\color{red}
	\overset{T}{\underset{s=1}{\sum}}
		\left(\,
			\overset{T}{\underset{t=1}{\prod}}\;
			P\!\left(\,D_{t} \,\left\vert\; S\overset{{\color{white}1}}{=}s,R_{e},R_{l}\right.\,\right)
			\right)
		\cdot
		P\!\left(\,S=s\,\right)
	}
\end{eqnarray*}
Hence,
\begin{eqnarray*}
\log\,P\!\left(\,D,R_{e},R_{l}\,\right)
& = &
	\log P\!\left(\,R_{e}\,\right)
	\; + \;
	\log P\!\left(\,R_{l}\,\right)
	\; + \;
	\log\left(\;
	\overset{T}{\underset{s=1}{\sum}}
		\left(\,
			\overset{T}{\underset{t=1}{\prod}}\;
			P\!\left(\,D_{t} \,\left\vert\; S\overset{{\color{white}1}}{=}s,R_{e},R_{l}\right.\,\right)
			\right)
		\cdot
		P\!\left(\,S=s\,\right)
	\,\right)
\end{eqnarray*}
Now, the last summand above can be rewritten as:
\begin{eqnarray*}
\log P\!\left(\,D \;\vert\, R_{e},R_{l}\,\right)
&=&
	\log\left\{\;\,
		\overset{T}{\underset{s=1}{\sum}}
		\left(\,
			\overset{T}{\underset{t=1}{\prod}}\;
			P\!\left(\,D_{t} \,\left\vert\; S\overset{{\color{white}1}}{=}s,R_{e},R_{l}\right.\,\right)
			\right)
		\cdot
		P\!\left(\,S=s\,\right)
		\,\right\}
\\
& = &
	\log\left\{\;\,
	\overset{T}{\underset{s=1}{\sum}}\;
	\exp\,\circ\,\log\left[\,
		\left(\,
			\overset{T}{\underset{t=1}{\prod}}\;
			P\!\left(\,D_{t} \,\left\vert\; S\overset{{\color{white}1}}{=}s,R_{e},R_{l}\right.\,\right)
			\right)
		\cdot
		P\!\left(\,S=s\,\right)
		\;\right]
	\,\right\}
\\
& = &
	\log\left\{\;\,
	\overset{T}{\underset{s=1}{\sum}}\;
	\exp\left[\;
		\log\left(\,
			\overset{T}{\underset{t=1}{\prod}}\;
			P\!\left(\,D_{t} \,\left\vert\; S\overset{{\color{white}1}}{=}s,R_{e},R_{l}\right.\,\right)
			\right)
		\, + \,
		\log P\!\left(\,S=s\,\right)
		\right]
	\,\right\}
\\
& = &
	\log\left\{\;\,
	\overset{T}{\underset{s=1}{\sum}}\;
	\exp\left[\;\log P\!\left(\,S=s\,\right)
			\, + \,
			\overset{T}{\underset{t=1}{\sum}}\,
				\log P\!\left(\,D_{t} \,\left\vert\; S\overset{{\color{white}1}}{=}s,R_{e},R_{l}\right.\,\right)
		\,\right]
	\,\right\}
\end{eqnarray*}

          %%%%% ~~~~~~~~~~~~~~~~~~~~ %%%%%

\noindent
Next, observe that:
\begin{eqnarray*}
&&
	\overset{T}{\underset{t=1}{\sum}}\;
	\log P\!\left(\,D_{t} \,\left\vert\; S\overset{{\color{white}1}}{=}s,R_{e},R_{l}\right.\,\right)
\;\; = \;\;
	\overset{s-1}{\underset{t=1}{\sum}}\;
	\log P\!\left(\,D_{t} \,\left\vert\; S\overset{{\color{white}1}}{=}s,R_{e},R_{l}\right.\,\right)
	\; + \;
	\overset{T}{\underset{t=s}{\sum}}\;
	\log P\!\left(\,D_{t} \,\left\vert\; S\overset{{\color{white}1}}{=}s,R_{e},R_{l}\right.\,\right)
\\
&=&
	\overset{s-1}{\underset{t=1}{\sum}}\;
	\log\textnormal{Poisson}\!\left(\,D_{t} \,\left\vert\; R_{e}\right.\,\right)
	\; + \;
	\overset{T}{\underset{t=s}{\sum}}
	\log\textnormal{Poisson}\!\left(\,D_{t} \,\left\vert\; R_{l}\right.\,\right)
\\
&=&
	{\color{white}+}\;\;\,
	\overset{s-1}{\underset{t=1}{\sum}}\;\,
	\log\textnormal{Poisson}\!\left(\,D_{t} \,\left\vert\; R_{e}\right.\,\right)
\\
&&
	+\;
	\overset{T}{\underset{t=s}{\sum}}
	\log\textnormal{Poisson}\!\left(\,D_{t} \,\left\vert\; R_{l}\right.\,\right)
	{\color{blue}
		\;+\;\,
		\overset{s-1}{\underset{t=1}{\sum}}\;
		\log\textnormal{Poisson}\!\left(\,D_{t} \,\left\vert\; R_{l}\right.\,\right)
		\;-\;
		\overset{s-1}{\underset{t=1}{\sum}}\;
		\log\textnormal{Poisson}\!\left(\,D_{t} \,\left\vert\; R_{l}\right.\,\right)
		}
\\
&=&
	\overset{s-1}{\underset{t=1}{\sum}}\;
	\log\textnormal{Poisson}\!\left(\,D_{t} \,\left\vert\; R_{e}\right.\,\right)
	\;+\;
	\overset{T}{\underset{t=1}{\sum}}\;
	\log\textnormal{Poisson}\!\left(\,D_{t} \,\left\vert\; R_{l}\right.\,\right)
	{\color{blue}
		\;-\;\,
		\overset{s-1}{\underset{t=1}{\sum}}\;
		\log\textnormal{Poisson}\!\left(\,D_{t} \,\left\vert\; R_{l}\right.\,\right)
		}
\\
&=&
	\overset{T}{\underset{t=1}{\sum}}\;
	\log\textnormal{Poisson}\!\left(\,D_{t} \,\left\vert\; R_{l}\right.\,\right)
	\;+\;
	\overset{s-1}{\underset{t=1}{\sum}}\;
	\log\textnormal{Poisson}\!\left(\,D_{t} \,\left\vert\; R_{e}\right.\,\right)
	{\color{blue}
		\;-\;\,
		\overset{s-1}{\underset{t=1}{\sum}}\;
		\log\textnormal{Poisson}\!\left(\,D_{t} \,\left\vert\; R_{l}\right.\,\right)
		}
\end{eqnarray*}

          %%%%% ~~~~~~~~~~~~~~~~~~~~ %%%%%

\noindent
Hence,
\begin{eqnarray*}
&&
	\log\left\{\;\,
	\overset{T}{\underset{s=1}{\sum}}\,
	{\color{red}
		\left(\,
			\overset{T}{\underset{t=1}{\prod}}\;
			P\!\left(\,D_{t} \,\left\vert\; S\overset{{\color{white}1}}{=}s,R_{e},R_{l}\right.\,\right)
			\right)
		\cdot
		P\!\left(\,S=s\,\right)
	}
	\,\right\}
\\
& = &
	\log\left\{\;
	\overset{T}{\underset{s=1}{\sum}}\;
	\exp\left[\;{\color{red}
			\log P\!\left(\,S=s\,\right)
			\, + \,
			\overset{T}{\underset{t=1}{\sum}}\,
				\log P\!\left(\,D_{t} \,\left\vert\; S\overset{{\color{white}1}}{=}s,R_{e},R_{l}\right.\,\right)
		}\,\right]
	\,\right\}
\\
&=&
	\log\left\{\;
	\overset{T}{\underset{s=1}{\sum}}\;
	\exp\left[\;
			\log P\!\left(\,S=s\,\right)
			\,+\,
			\overset{T}{\underset{t=1}{\sum}}\;
			\log\textnormal{Poisson}\!\left(\,D_{t} \,\left\vert\; R_{l}\right.\,\right)
			\,+\,
			\overset{s-1}{\underset{t=1}{\sum}}\;
			\log\textnormal{Poisson}\!\left(\,D_{t} \,\left\vert\; R_{e}\right.\,\right)
			\,-\,
			\overset{s-1}{\underset{t=1}{\sum}}\;
			\log\textnormal{Poisson}\!\left(\,D_{t} \,\left\vert\; R_{l}\right.\,\right)
		\,\right]
	\right\}
\\
&=&
	\texttt{log\_sum\_exp}
	\left[\;
		\log P\!\left(\,S=s\,\right)
		\,+\,
		\overset{T}{\underset{t=1}{\sum}}\;
		\log\textnormal{Poisson}\!\left(\,D_{t} \,\left\vert\; R_{l}\right.\,\right)
		\,+\,
		\overset{s-1}{\underset{t=1}{\sum}}\;
		\log\textnormal{Poisson}\!\left(\,D_{t} \,\left\vert\; R_{e}\right.\,\right)
		\,-\,
		\overset{s-1}{\underset{t=1}{\sum}}\;
		\log\textnormal{Poisson}\!\left(\,D_{t} \,\left\vert\; R_{l}\right.\,\right)
		\,\right]
\end{eqnarray*}

          %%%%% ~~~~~~~~~~~~~~~~~~~~ %%%%%


          %%%%% ~~~~~~~~~~~~~~~~~~~~ %%%%%
